\documentclass{article}
\usepackage[fontset=ubuntu]{ctex}
\usepackage{amsmath}
\usepackage{amssymb}

\begin{document}

\setcounter{section}{3}

\section{Question 4}
\subsection{Question 4-a}

\emph{Question} Given tn is the data point of the data set $D=\{t_N, \ldots, t_N\}$. Each data point $t_n$ is associated with a weighting factor $r_n > 0$.
The sum-of-squares error function becomes:
\begin{equation*}
E_D(\textbf w)=\frac{1}{2}\sum\limits_{n=1}^N r_n(t_n−\mathbf w^{\mathbf T} \mathbf x_n)^2
\end{equation*}

Find the solution $\mathbf w^∗$ that minimizes the error function.

\emph{Answer} 假設$\mathbf w$是一個$k$維的向量$(w_1, w_2, \ldots, w_k)$。將$E_D(\mathbf w)$對$\mathbf w$的第$i$個分量$w_i$偏微分得到:

\begin{equation}\label{differential}
\frac{\partial}{\partial w_i}E_D(\mathbf w)
=\frac{1}{2}\cdot2\sum\limits_{n=1}^N r_n(t_n-\mathbf w^{\mathbf T} \mathbf x_n)(-x_{ni})
\end{equation}

方程式\ref{differential}中,$x_{ni}$表示向量$\mathbf x_n$的第$i$個分量。

為求極值,令$\frac{\partial}{\partial w_i}E_D(\mathbf w)$在$w_i ^*$的值為0。因此:

\begin{equation}
0=\sum\limits_{n=1}^N r_n(t_n-\mathbf w^{*\mathbf T} \mathbf x_n)(-x_{ni})
\end{equation}

移項並取轉置得到:

\begin{equation}
\begin{aligned}
\sum\limits_{n=1}^N r_n t_n x_{ni}
=& \sum\limits_{n=1}^N r_n(\mathbf w^{*\mathbf T} \mathbf x_n)x_{ni} \\
=& \sum\limits_{n=1}^N r_n x_{ni} ( \mathbf x_n^{\mathbf T} \mathbf w^*)
\end{aligned}
\end{equation}

將所有分量合併回向量:

\begin{equation}\label{almost}
\sum\limits_{n=1}^N r_n t_n \mathbf x_n
= \sum\limits_{n=1}^N r_n \mathbf x_n \mathbf x_n^{\mathbf T} \mathbf w^*
\end{equation}

方程式\ref{almost}中等號右邊的$\mathbf w^*$與$n$無關,餘下的項為一個方陣。故:

\begin{equation}\label{minimum}
\mathbf w^*=(\sum\limits_{n=1}^N r_n \mathbf x_n \mathbf x_n^{\mathbf T})^{-1}\sum\limits_{n=1}^N r_n t_n \mathbf x_n
\end{equation}

$E_D(\mathbf w)$對$w_i$二次偏微分的結果為:

\begin{equation}
\frac{\partial^2}{\partial w_i^2}E_D(\mathbf w)
=\sum\limits_{n=1}^N r_n x_{ni}^2
\end{equation}

此值為正,因此方程式\ref{minimum}中所列極值為最小值。

\subsection{Question 4-b}

\emph{Question} Following the previous problem (4-a), if
\begin{equation*}
\mathbf t = [t_1 t_2 t_3] = \begin{bmatrix}
    0&10&5
\end{bmatrix},
\mathbf X=[\mathbf{x_1 x_2 x_3}] = \begin{bmatrix}
    2 & 5 & 5 \\[0.3em]
    3 & 1 & 6 \\[0.3em]
\end{bmatrix}
\end{equation*}

\begin{equation*}
r_1 = 2, r_2 = 1, r_3 = 3
\end{equation*}

\section{Question 5}
\emph{Question} Given a linear model:

\begin{equation*}
y(x, \mathbf w) = w_0 + \sum_{i=1}^{D}w_i x_i
\end{equation*}

with a sum-of-squares error function:

\begin{equation*}
E(\mathbf w) = \frac 1 2 \sum_{n=1}^{N} \big(y(x_n, \mathbf w) -t_n ) \big)^2
\end{equation*}

where $t_n$ is the data point of the data set $\mathcal D=\{t_1, \ldots,t_N \}$

Suppose that Gaussian noise $\epsilon_i$ with zero mean and variance $\sigma^2$ is added independently to each of the input variables $x_i$.
By making use of $\mathbb E[\epsilon_i \epsilon_j] = \delta_{ij} \sigma^2$ and $\mathbb E[\epsilon_i] = 0$ show that minimizing $E$ averaged over the noise distribution is equivalent to minimizing the sum-of-squares error for noise-free input variables with the addition of a weight-decay regularization term, in which the bias parameter $w_0$ is omitted from the regularizer.

\emph{Answer} $E(\mathbf w)$對於$\epsilon_i$的平均為其期望值:

\begin{equation}
\begin{aligned}
\mathbb E[E(\mathbf w)]
=& \frac{1}{2}\mathbb E\left[\sum\limits_{n=1}^N (y(x_n, \mathbf w)-t_n)^2\right]\\
=& \frac{1}{2}\mathbb E\left[\sum\limits_{n=1}^N \left(w_0+\sum\limits_{i=1}^D w_i(x_{ni}+\epsilon_i)-t_n\right)^2\right]\\
=& \frac{1}{2}\mathbb E\left[\sum\limits_{n=1}^N (w_0-t_n)^2+2(w_0-t_n)\sum\limits_{i=1}^D w_i(x_{ni}+\epsilon_i)+\left(\sum\limits_{i=1}^D w_i(x_{ni}+\epsilon_i)\right)^2\right]\\
=& \frac{1}{2}\sum\limits_{n=1}^N (w_0-t_n)^2+2(w_0-t_n)\sum\limits_{i=1}^D w_i x_{ni}+\mathbb E\left[\left(\sum\limits_{i=1}^D w_i(x_{ni}+\epsilon_i)\right)^2\right]\\
\end{aligned}
\end{equation}

第三項進一步展開:

\begin{equation}
\begin{aligned}
& \mathbb E\left[\left(\sum\limits_{i=1}^D w_i(x_{ni}+\epsilon_i)\right)^2\right]\\
=& \sum\limits_{i=1}^D w_i^2 x_{ni}^2 + w_i^2 \mathbb E[\epsilon_i^2] + 2\sum\limits_{i=1}^D \sum\limits_{\genfrac{}{}{0pt}{2}{j=1}{j\neq i}}^D w_i w_j \mathbb E[(x_{ni}+\epsilon_i)(x_{nj}+\epsilon_j)] \\
=& \sum\limits_{i=1}^D w_i^2 x_{ni}^2 + w_i^2 \sigma^2 + 2\sum\limits_{i=1}^D \sum\limits_{\genfrac{}{}{0pt}{2}{j=1}{j\neq i}}^D w_i w_j x_{ni} x_{nj} \\
=& \sum\limits_{i=1}^D (w_i x_{ni} + w_i)^2 + w_i^2(\sigma^2-1)
\end{aligned}
\end{equation}

$\mathbb E[E(\mathbf w)]$對$w$的第$k$個分量$w_k$的偏微分為:

\begin{equation}
\frac{\partial}{\partial w_k}\mathbb E[E(\mathbf w)] = \sum\limits_{n=1}^N 2(w_0-t_n)x_{nk}+2\sum\limits_{i=1}^D(w_i x_{ik} + w_i) x_{nk} + w_k^2 (\sigma^2-1)
\end{equation}

\section{Question 6}

\emph{Question} $\mathbf A \in \mathbb R^{n \times n}, \alpha$, $\alpha$ is one of the elements of $\mathbf A$, prove that

\begin{equation*}
\frac{\mathrm d}{\mathrm d \alpha }\ln|\mathbf A|= \rm Tr\bigg(\mathbf A^{-1}\frac{\mathrm d}{\mathrm d \alpha}\mathbf A \bigg)
\end{equation*}
where the matrix $\mathbf A$ is a real, symmetric, non-singular matrix.

\emph{Answer}

\end{document}
